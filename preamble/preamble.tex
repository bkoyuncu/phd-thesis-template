% ===================================================================
% GENERAL PREAMBLE

\documentclass[ a4paper,
				fontsize=11pt,
				twoside=true,
				numbers=noenddot,
                open=any,
                secnumdepth=2,
%				draft % Shows black boxes next to overful hboxes (ignore the errors)
			   ]{preamble/thesis}

%\usepackage[main=british,ngerman,spanish]{babel} 		% English language/hyphenation
\usepackage[main=british]{babel} 		% English language/hyphenation
\usepackage[dissertation, internal]{preamble/udstitle}

\usepackage[T1]{fontenc}

% ===================================================================
% COLORS

\usepackage{xcolor} 				% To define colors
% ===================================================================
% SAARLAND UNIVERSITY

\definecolor{SBred}{RGB}{200,34,84}
\definecolor{SBdark}{RGB}{1,40,63}
%\definecolor{SBgold}{RGB}{215,223,35}
\definecolor{SBgold}{RGB}{171,179,25}  % darker than the official
\definecolor{SBblue}{RGB}{0,72,119}

\colorlet{darkgreen}{green!50!black}

%%% Local Variables:
%%% mode: latex
%%% TeX-master: "../thesis"
%%% End:
			% Pre-define a bunch of often-used colors

\colorlet{maincolor}{SBblue}		% Main color, used for headings, etc.
\colorlet{secondcolor}{SBred}		% Secondary color, used for references, etc.
\colorlet{thirdcolor}{SBgold}		% Tertiary color, used for citations, etc.
\colorlet{fourthcolor}{SBgold}		% Fourth color, used for URLs, etc.
\colorlet{lightgraycolor}{SBdark!50}% A gray color, used for lines, etc.
\colorlet{darkgraycolor}{SBdark!80}	% A dark color, used for numbers, etc.

% ===================================================================
% GRAPHICS

\usepackage{pifont}					% For special symbols like checkmarks
\usepackage[labelsep=colon]{caption}				% Options for caption
\DeclareCaptionFont{labelcolor}{\color{maincolor}}
\captionsetup{labelfont={labelcolor,bf}} 	% e.g. coloring the word "Figure" in caption
\usepackage{subcaption}				% For subcaptioning images
\usepackage{fancybox}				% Allows shadowbox around figure (for screenshot)
\usepackage{fontawesome}
\usepackage{standalone}

\graphicspath{
    {figures}
}


% Tikz
\usetikzlibrary{
  shapes,
  pgfplots.groupplots,
  shadings,
  calc,
  arrows,
  backgrounds,
  colorbrewer,
  shadows.blur,
  external,
  shapes,
  shapes.arrows,
  shapes.symbols,
  arrows.meta,
  fit % For extending the bounding box
}

% PGFPlot
\usepackage{pgfplots}
\usepgfplotslibrary{groupplots,fillbetween}
\pgfplotsset{compat=1.15}

% Externalize (disabled by default)
\usepgfplotslibrary{external}
\tikzexternalize[prefix=tikz/, mode=list and make]
\tikzexternaldisable

% Create new lenght for pgfplots
\newlength\figureheight
\newlength\figurewidth
\setlength\figureheight{\textheight}
\setlength\figurewidth{\textwidth}

% width in figure* environments
\newlength{\thesiswidewidth}
\setlength{\thesiswidewidth}{457.8024pt}

\pdfsuppresswarningpagegroup=1		% Supress the "PDF inclusion: multiple pdfs with page group..." warning (https://tex.stackexchange.com/questions/76273/multiple-pdfs-with-page-group-included-in-a-single-page-warning)

% Customized shadow color for shadowbox
\makeatletter
\newcommand\Cshadowbox{\VerbBox\@Cshadowbox}
\def\@Cshadowbox#1{%
	\setbox\@fancybox\hbox{\fbox{#1}}%
	\leavevmode\vbox{%
		\offinterlineskip
		\dimen@=\shadowsize
		\advance\dimen@ .5\fboxrule
		\hbox{\copy\@fancybox\kern.5\fboxrule\lower\shadowsize\hbox{%
				\color{lightgraycolor}\vrule \@height\ht\@fancybox \@depth\dp\@fancybox \@width\dimen@}}%
		\vskip\dimexpr-\dimen@+0.5\fboxrule\relax
		\moveright\shadowsize\vbox{%
			\color{lightgraycolor}\hrule \@width\wd\@fancybox \@height\dimen@}}}
\makeatother

% ===================================================================
% FONTS

\usepackage[defaultsans]{lato}		% Sans Font: Lato (would allow for thin)
%\usepackage{gillius}				% Sans Font: Tufte font

\usepackage{anyfontsize}			% Allows arbitrary font sizes

% Headings should use sans serif font
\addtokomafont{part}{\normalfont\sffamily\color{maincolor}}

% Colored chapter number but not the chapter title
\addtokomafont{chapter}{\normalfont\sffamily}
\addtokomafont{chapterprefix}{\normalfont\sffamily\color{maincolor}}
\addtokomafont{section}{\normalfont\sffamily\color{maincolor}}
\addtokomafont{subsection}{\normalfont\sffamily\color{maincolor}}
\addtokomafont{subsubsection}{\normalfont\sffamily\color{maincolor}}

\addtokomafont{pagehead}{\normalfont\sffamily}	% Also use sans serif font in header

% Checkmarks
\newcommand{\cmark}{\ding{51}}%
\newcommand{\xmark}{\ding{55}}%
\newcommand{\minusmark}{\text{-}} % minus mark

% Use Computer Modern Symbols for mathcal
\DeclareMathAlphabet{\mathcal}{OMS}{cmsy}{m}{n}
\SetMathAlphabet{\mathcal}{bold}{OMS}{cmsy}{b}{n}

% Fixes overful hboxes in references
% see https://tex.stackexchange.com/questions/171999/overfull-hbox-in-biblatex
\emergencystretch=1em


% ===================================================================
% REFERENCES
% ===================================================================

\usepackage{varioref}
\usepackage{hyperref}
\usepackage[capitalize,nameinlink,noabbrev]{cleveref}
\usepackage{epigraph}


% Customization of hyperref options
\hypersetup{
	unicode, % Use unicode for links
	pdfborder={0 0 0}, % Suppress border around pdf
	bookmarksdepth=section,
	bookmarksopen=true, % Expand the bookmarks as soon as the pdf file is opened
	%bookmarksopenlevel=4,
	%	linktoc=all, % Toc entries and numbers links to pages
	linktocpage=true,  % Only the page number links to pages
	breaklinks=true,
	colorlinks=true,
	citecolor = thirdcolor,
	linkcolor = secondcolor,
	urlcolor = fourthcolor,
	final, % so that we have links in draft mode
}


%----------------------------------------------------------------------------------------
%	CLEVEREF CUSTOMISATION
%----------------------------------------------------------------------------------------

% Hyperlink the page reference as well
\let\oldvpageref\vpageref
\renewcommand{\vpageref}[1]{\hyperref[#1]{\oldvpageref{#1}}}

% Remove parentheses around equations
\creflabelformat{equation}{#2\textup{#1}#3}

% TODO not sure if I like this one
%% Set the refname for subsections
\newcommand{\subsectionname}{Subsection}
\newcommand{\subsectionnameplural}{Subsections}
\crefname{subsection}{\subsectionname}{\subsectionname}
\Crefname{subsection}{\subsectionname}{\subsectionnameplural}

%% Set the refname for side notes
\crefname{sidenote}{side note}{side notes}
\Crefname{sidenote}{Side note}{Side notes}

% ===================================================================
% MATH
% ===================================================================

% Fix "Too many math alphabets" error (https://tex.stackexchange.com/a/243541)
\newcommand\hmmax{0}
\newcommand\bmmax{0}

\usepackage{amsfonts}				% blackboard math symbols
\usepackage{amsmath}				% for all basic math operations
\usepackage{interval}				% for typesetting intervals
\usepackage{amssymb}
\usepackage{dsfont}					% indicator function
\usepackage{nicefrac}				% compact symbols for 1/2, etc.
\usepackage{cancel}					% To "cancel" numbers in equations
%\usepackage{ntheorem}				% required by kaobook (otherwise \theoremstyle is not defined)
\usepackage[
    separate-uncertainty=true, 
    output-exponent-marker = \text{e},
    exponent-product={},
    retain-zero-uncertainty=true
]{siunitx}  						% SI units (used for run times)
\usepackage{xfrac}					% For slanted fractions "sfrac" like the nicefrac
\usepackage{bm}						% Offers \bm to create bold math (used by Goodfellow)
\usepackage{derivative}				% Handy commands for typesetting derivatives
\usepackage{mathtools}				% for vcentcolon a centered colon used in \eqdef
\usepackage{etoolbox}
% Load mathematical packages for theorems and related environments
\usepackage[framed=true,background=maincolor!5!white]{preamble/kaotheorems}

\usepackage{annotate-equations}		% https://ctan.org/pkg/annotate-equations?lang=en
\renewcommand{\eqnhighlightheight}{\vphantom{\hat{H}_{\theta}}\mathstrut} % (try to) make color boxes of equal height


%%%%%%%%%%%%%%%%%%%%%%%%%%%%%%%%%%%%%%%%%%%%%%%%%%%
%% SIUNITX CONFIGURATION
%%%%%%%%%%%%%%%%%%%%%%%%%%%%%%%%%%%%%%%%%%%%%%%%%%%
\sisetup{product-units=single}  % so that the units in a product only appear at the end 

\DeclareSIUnit{\microsecond}{\SIUnitSymbolMicro s}  % I am not sure this is necessary
\DeclareSIUnit{\px}{px}  % pixels
\DeclareSIUnit{\million}{M}

% Bugfix: https://tex.stackexchange.com/questions/262142/thmtools-notebraces-bug
%\makeatletter
%%%% from amsthm.sty
%\def\thmhead@plain#1#2#3{%
%	\thmname{#1}\thmnumber{\@ifnotempty{#1}{ }\@upn{#2}}%
%	%%% the line below had (##3)
%	\thmnote{ {\the\thm@notefont\thm@lparen#3\thm@rparen}}}
%
%%%% from thm-amsthm.sty
%\def\thmt@setheadstyle#1{%
%	\thmt@style@headstyle{%
%		\def\NAME{\the\thm@headfont ##1}%
%		\def\NUMBER{\bgroup\@upn{##2}\egroup}%
%		%%% the line below had (##3)
%		\def\NOTE{\if=##3=\else\bgroup\thmt@space\the\thm@notefont\thm@lparen##3\thm@rparen\egroup\fi}%
%	}%
%	\def\thmt@tmp{#1}%
%	\@onelevel@sanitize\thmt@tmp
%	%\tracingall
%	\ifcsname thmt@headstyle@\thmt@tmp\endcsname
%	\thmt@style@headstyle\@xa{%
%		\the\thmt@style@headstyle
%		\csname thmt@headstyle@#1\endcsname
%	}%
%	\else
%	\thmt@style@headstyle\@xa{%
%		\the\thmt@style@headstyle
%		#1%
%	}%
%	\fi
%	%\showthe\thmt@style@headstyle
%}
%%%% the line below had (#3)
%\def\thmt@embrace#1#2\thm@lparen#3\thm@rparen{#1#3#2}
%%%% added for default
%\def\thm@lparen{(}\def\thm@rparen{)}
%\makeatother

\xspaceaddexceptions{]}

% \declaretheoremstyle[
% %spaceabove=.5\thm@preskip,
% %spacebelow=.5\thm@postskip,
% %headfont=\normalfont\bfseries,%\scshape,
% notefont=\bfseries,
% notebraces={ [}{]},
% bodyfont=\normalfont,
% %headformat={\NAME\space\NUMBER\space\NOTE},
% headpunct={},
% postheadspace=\newline,
% %prefoothook={\hfill\qedsymbol}
% ]{mytheoremstyle}

% \theoremstyle{mytheoremstyle}
% \declaretheorem[
% name=Definition,
% %refname={definition,definitions},
% refname={Definition,Definitions},
% Refname={Definition,Definitions},
% numberwithin=section,
% mdframed={
% 	style=mdfkao,
% 	backgroundcolor=lightgraycolor!25,
% 	%frametitlebackgroundcolor=\@theorembackground,
% },
% ]{thesisdefinition}


% \declaretheorem[name=Update Rule,
% refname={update rule,update rules},
% Refname={Update Rule,Update Rules},
% numberwithin=section,
% mdframed={
% 	style=mdfkao,
% 	backgroundcolor=lightgraycolor!25,
% %	%frametitlebackgroundcolor=\@theorembackground,
% },
% sibling=thesisdefinition]{thesisupdaterule}

% \declaretheorem[
% name=Theorem,
% refname={Theorem,Theorems},
% Refname={Theorem,Theorems},
% numberwithin=section,
% mdframed={
% 	style=mdfkao,
% 	backgroundcolor=lightgraycolor!25,
% 	%	%frametitlebackgroundcolor=\@theorembackground,
% },
% sibling=thesisdefinition
% ]{thesistheorem}

% ===================================================================
% CODE
% !!! Important to keep this after the definition of fancybox package
% 	  as it otherwise will create a weird bug !!!

% Inline code that looks similar to markdown inline code snippets
\newcommand{\inlinecode}[1]{%
	\begin{tikzpicture}[baseline=0ex]%
		\node[anchor=base,%
		text height=1em,%
		text depth=1ex,%
		inner ysep=0pt,%
		draw=darkgraycolor!30!white,%
		fill=lightgraycolor!20!white,%
		rounded corners=2pt] at (0,0) {\footnotesize\texttt{#1}};%
	\end{tikzpicture}%
}
% TikZ code contains fragile commands which throws errors when used in captions
% and section titles. This robust command can be used as drop-in
% (https://tex.stackexchange.com/a/56081)
\DeclareRobustCommand\robustInlinecode[1]{\inlinecode{#1}}

\usepackage{listings}				% For full code blocks
\usepackage{lstautogobble}

\input{preamble/lstlinebgrd.tex}	% Include lstlinebgrd (with some fixes) for highlighting code lines

\lstdefinestyle{thesisstyle}{
	backgroundcolor=\color{maincolor!5!white},
	commentstyle=\bfseries\itshape\color{maincolor},
	keywordstyle=\bfseries\color{maincolor},
	numberstyle=\tiny\color{maincolor},
	stringstyle=\bfseries\color{thirdcolor},
	basicstyle=\ttfamily\footnotesize,
    xleftmargin=3.2ex,
	breakatwhitespace=false,
	breaklines=true,
	captionpos=t,
	keepspaces=true,
	numbers=left,
	numbersep=7pt,
	showspaces=false,
	showstringspaces=false,
	showtabs=false,
	tabsize=4,
    escapeinside={(@}{@)},
    rulecolor=\color{maincolor},
    print,
}
\lstset{
    style=thesisstyle, 
    xleftmargin=.04\textwidth,
    autogobble=true,
}

% Define an algorithm environment using the listings package

\newcounter{nalg}[chapter] % defines algorithm counter for chapter-level
\renewcommand{\thenalg}{\thechapter .\arabic{nalg}} %defines appearance of the algorithm counter
\DeclareCaptionLabelFormat{algocaption}{Algorithm \thenalg} % defines a new caption label as Algorithm x.y

\lstnewenvironment{lstalgorithm}[1][] %defines the algorithm listing environment
{   
    \refstepcounter{nalg} %increments algorithm number
    \captionsetup{labelformat=algocaption,labelsep=colon} %defines the caption setup for: it ises label format as the declared caption label above and makes label and caption text to be separated by a ':'
    \lstset{ %this is the style
        mathescape=true,
        numbers=left, 
        numberstyle=\tiny,
%        basicstyle=\scriptsize, % TODO need to decide on this one
        keywordstyle=\color{black}\bfseries\em,
        keywords={,input, output, return, for, in, if, else, while, begin, end, do, done, function,} %add the keywords you want, or load a language as Rubens explains in his comment above.
        numbers=left,
        style=thesisstyle,
        autogobble=true,
        morecomment=[l]{\#},
        escapechar=\&,
        framerule=0.5pt,
        #1 % this is to add specific settings to an usage of this environment (for instance, the caption and referable label)
    }
}
{}

%\newtcbinputlisting{xmlListing}{colframe=background, colback=background, listing options={language=xml,columns=fullflexible},listing only }

\lstdefinestyle{algorithmstyle}{
    backgroundcolor={},
    mathescape=true,
    numbers=left, 
    numberstyle=\tiny,
    %        basicstyle=\scriptsize, % TODO need to decide on this one
    keywordstyle=\color{black}\bfseries\em,
    keywords={,input, return, for, in, if, else, while, begin, end, do, done, function,} %add the keywords you want, or load a language as Rubens explains in his comment above.
    numbers=left,
    style=thesisstyle,
    morecomment=[l]{\#},
    escapechar=\&,
    framerule=0pt,
}
% TODO Clean up above

\usepackage{tcolorbox}
\usepackage{moreverb}

\newenvironment{algorithmfloat}{%
    \table[!hbtp]
    \refstepcounter{nalg} %increments algorithm number
    \captionsetup{labelformat=algocaption,labelsep=colon} %defines the caption setup for: it ises label format as the declared caption label above and makes label and caption text to be separated by a ':'
}{%
    \endtable
}

\newcommand{\readalgorithm}[1]{%
    \begin{mdframed}[%
        innerrightmargin=0pt, innerbottommargin=0pt, 
        innerleftmargin=\dimexpr-7pt-1.9pt, innertopmargin=\dimexpr0.25\ht\strutbox,
        userdefinedwidth=\dimexpr\linewidth-7pt-1.9pt,
        linewidth=0.5pt, linecolor=maincolor, topline=false, bottomline=false, rightline=false,
        backgroundcolor=maincolor!5!white
        ]
        \lstinputlisting[style=algorithmstyle]{#1}
    \end{mdframed}
}

\crefname{nalg}{Algorithm}{Algorithm}

% ===================================================================
% CITATION

% CHANGE IN THE KAOBILBIO STYLE!!!
% DO NOT CLEARFIELD FOR archivePrefix, arxivId, and eprint!!!
% NEED TO DO THIS IF UPDATE THE KAOBOOK TEMPLATE!!!

\usepackage[
	bibstyle=numeric,			% Use numeric citations in bibliography, e.g. [5]
	citestyle=numeric-comp,		% Use compact numeric citations, e.g. [1-5,7]
	sorting=nyt,				% Sort Bibliography by name then year then title (alternative: none=citation order)
	maxnames=99,				% Show a maximum of 99 author names in Bibliography
	mincitenames=1,				% Show at least two author names when citing ...
	maxcitenames=2,				% ... But never more than two
	sortcites=true,				% Automatically sort citations numerically, e.g. [5,1,3] -> [1,3,5]
	date=year,					% Printed dates only show year
	abbreviate=false,			% Don't abbreviate string such as editor -> ed. or Tech. rep.
	% Hide some information generally:
	isbn=false,
	doi=false,
	related=false,
]{preamble/kaobiblio}

\usepackage{xpatch}				% Patch to customize the look of the Bibliography

\addbibresource{bibliography/bibliography.bib}	% Bibliography file

% https://tex.stackexchange.com/questions/99316/symbol-for-external-links
\newcommand{\ExternalLink}{%
    \tikz[x=1.2ex, y=1.2ex, baseline=-0.05ex]{% 
        \begin{scope}[x=1ex, y=1ex]
            \clip (-0.1,-0.1)
            --++ (-0, 1.2)
            --++ (0.6, 0)
            --++ (0, -0.6)
            --++ (0.6, 0)
            --++ (0, -1);
            \path[draw,
            line width = 0.9,
            rounded corners=0.5]
            (0,0) rectangle (1,1);
        \end{scope}
        \path[draw, line width = 0.9] (0.5, 0.5)
        -- (1, 1);
        \path[draw, line width = 0.9] (0.6, 1)
        -- (1, 1) -- (1, 0.6);
    }
}
\DeclareFieldFormat{citeurl}{\href{#1}{\small\ExternalLink}}
\newrobustcmd*{\citefirstlastauthor}{\AtNextCite{\DeclareNameAlias{labelname}{given-family}}\citeauthor}


% Customize what appears in the margin citation (removed the ":")
\renewcommand{\formatmargincitation}[1]{%
	\parencite{#1} \citeauthor*{#1} (\citeyear{#1}), \citetitle{#1}\xspace\citeurl{#1}%
}

% A custom command for only creating a marginnote with the citation
\newcommand{\onlysidecite}[2][]{\marginnote[#1]{%
	\parencite{#2} \citeauthor*{#2} (\citeyear{#2}), \citetitle{#2}%
}}

\renewbibmacro{in:}{}			% Remove "in:" from Bibliography

% Put quotes around the title of misc entries (e.g. arXiv papers) similar to "regular" paper
\DeclareFieldFormat[article,inproceedings,misc,book,techreport]{citetitle}{\mkbibquote{#1\addperiod}}
\DeclareFieldFormat[article,inproceedings,misc,book,techreport]{title}{\mkbibquote{#1\addperiod}}

% Publisher in Book in italics (similar to paper)
\DeclareListFormat{publisher}{%
	\usebibmacro{list:delim}{#1}%
	\mkbibemph{#1\isdot}
	\usebibmacro{list:andothers}}

% Currently this is missing a period to seperate it from the rest!
%% Remove paranthesis around the date of article
%\renewbibmacro*{issue+date}{%
%	\printfield{issue}%
%	\setunit*{\addspace}%
%	\usebibmacro{date}%
%	\newunit}


% from https://tex.stackexchange.com/questions/468623/indicating-joint-first-authorship-through-special-markup-in-biblatex-biber
\newcommand{\mkbibcoauthor}[1]{%
    \ifitemannotation{jointfirst}%
    {{#1}*}%   e.g., J͟o͟h͟n͟_D͟o͟e͟* (if jointfirst)
    {#1}%                e.g., John Doe  (otherwise)
}


% This command tell biblatex to bold an author
% Taken from here: https://tex.stackexchange.com/questions/33330/make-one-authors-name-bold-every-time-it-shows-up-in-the-bibliography
\newcommand{\makeauthorbold}[1]{%
  \DeclareNameFormat{author}{%
    \ifthenelse{\value{listcount}=1}
    {%
      {\expandafter\ifstrequal\expandafter{\namepartfamily}{#1}{\mkbibbold{\mkbibcoauthor{\namepartfamily\addcomma\addspace \namepartgiveni}}}{\mkbibcoauthor{\namepartfamily\addcomma\addspace \namepartgiveni}}}
      %
    }{\ifnumless{\value{listcount}}{\value{liststop}}
        {\expandafter\ifstrequal\expandafter{\namepartfamily}{#1}{\mkbibbold{\mkbibcoauthor{\addcomma\addspace\namepartfamily\addcomma\addspace\namepartgiveni}}}{\mkbibcoauthor{\addcomma\addspace\namepartfamily\addcomma\addspace\namepartgiveni}}}
        {\expandafter\ifstrequal\expandafter{\namepartfamily}{#1}{\mkbibbold{\mkbibcoauthor{\addcomma\addspace and\addspace\namepartfamily\addcomma\addspace\namepartgiveni\addcomma\isdot}}}{\mkbibcoauthor{\addcomma\addspace and\addspace\namepartfamily\addcomma\addspace\namepartgiveni\addcomma\isdot}}}%
      }
    \ifthenelse{\value{listcount}<\value{liststop}}{\addcomma\space}{} % TODO add "and" 
%    \ifthenelse{\value{listcount}=\numexpr\value{liststop}-1\relax}{and }{}
  }
}

\makeauthorbold{Javaloy}


% ===================================================================
% GLOSSARIES

% Example 21: One hyper link per entry per chapter
\glsaddstoragekey{chapter}{0}{\glschapnum}

%\renewcommand*{\glslinkpostsetkeys}{%
%    \edef\currentchap{\arabic{chapter}}%
%    \ifnum\currentchap=\glschapnum{\glslabel}\relax
%    \setkeys{glslink}{hyper=false}%
%    \else
%    \glsifhyperon{\glsfieldxdef{\glslabel}{chapter}{\currentchap}}%
%    \fi
%}

% Changes the hyper link color after first use per chapter
\makeatletter
\newcommand*{\glsplainhyperlink}[2]{%
    \begingroup%
    \edef\currentchap{\arabic{chapter}}%
    \ifnum\currentchap=\glschapnum{\glslabel}\relax%
        \hypersetup{hidelinks}%
        \hyperlink{#1}{#2}%
    \else%
%        \glsfieldxdef{\glslabel}{chapter}{\currentchap}%
        \hyperlink{#1}{#2}%
    \fi%
    \endgroup%
}
\let\@glslink\glsplainhyperlink

\renewcommand{\glspostlinkhook}{%
    \edef\currentchap{\arabic{chapter}}%
    \ifnum\currentchap=\glschapnum{\glslabel}\relax%
    \else%
        \glsfieldxdef{\glslabel}{chapter}{\currentchap}%
    \fi%
}

\let\oldifglsused\ifglsused
\renewcommand*{\ifglsused}[3]{%
    \edef\currentchap{\arabic{chapter}}%
    \ifnum\currentchap=\glschapnum{\glslabel}\relax%
        #2%
    \else%
        #3%
    \fi%
}

\AtEndDocument{%
    % Glossaries uses the old command to decide if write the prevcount to the aux file
    \let\ifglsused\oldifglsused% 
}
    
\makeatother


% Enable cgls commands
\glsenableentrycount

% Custom glossary style
% TODO this one or 'super'?
% Reference in /usr/share/texlive/texmf-dist/tex/latex/glossaries/styles
\newglossarystyle{myglossarystyle}{%
    \setglossarystyle{tree}% base this style on the list style
    
    % make nothing happen between groups:
    \renewcommand{\glsgroupskip}{}%
    \renewcommand{\glossentry}[2]{%
        \hangindent0pt\relax
        \parindent0pt\relax
        \glsentryitem{##1}\glstreenamefmt{\glstarget{##1}{\glossentryname{##1}}}%
        \ifglshassymbol{##1}{\space(\glossentrysymbol{##1})}{}%
        \glstreepredesc\Glossentrydesc{##1}\glspostdescription\space (Pages ##2)\par
    }%
    
}

% ===================================================================
% TABLES

\usepackage{tabularx}			% Tables with flexible column width
\usepackage{multirow}			% Allow cells spread over multiple rows
\usepackage{bigdelim}			% Defines big parentheses/brackets/etc for tables
\usepackage{colortbl}			% define BG colors of cells via \cellcolor
\usepackage{makecell}			% Multi-lined tabular cells (used in DeepOBS results table)

% Customization of makecell package
\renewcommand{\cellalign}{tl}
\renewcommand\theadalign{bc}
\renewcommand\theadfont{\bfseries}
\renewcommand\theadgape{\Gape[4pt]}
\renewcommand\cellgape{\Gape[4pt]}

% New column types for tables with mathmode
\usepackage{array}   % for \newcolumntype macro
\newcolumntype{L}{>{$}l<{$}} % math-mode version of "l" column type
\newcolumntype{C}{>{$}c<{$}} % math-mode version of "c" column type
\newcolumntype{R}{>{$}r<{$}} % math-mode version of "r" column type

% ===================================================================
% TODONOTES

\usepackage{blindtext}
\usepackage{lipsum}
\usepackage{xargs}				% Use more than one optional parameter in a new commands

\PassOptionsToPackage{colorinlistoftodos,prependcaption}{todonotes}
\newcommandx{\unsure}[2][1=]{\todo[linecolor=secondcolor,backgroundcolor=secondcolor!25,bordercolor=secondcolor,size=\footnotesize,#1]{\textbf{Unsure:}\xspace#2}}
\newcommandx{\change}[2][1=]{\todo[linecolor=maincolor,backgroundcolor=maincolor!25,bordercolor=maincolor,size=\footnotesize,#1]{\textbf{Change:}\xspace#2}}
\newcommandx{\info}[2][1=]{\todo[linecolor=darkgraycolor,backgroundcolor=darkgraycolor!25,bordercolor=darkgraycolor,size=\footnotesize,#1]{\textbf{Info:}\xspace#2}}
\newcommandx{\improvement}[2][1=]{\todo[linecolor=thirdcolor,backgroundcolor=thirdcolor!25,bordercolor=thirdcolor,size=\footnotesize,#1]{\textbf{Improve:}\xspace#2}}

% ===================================================================
% CUSTOMIZATIONS

% Only show sections in margin TOC (don't show subsections, etc.)
\setcounter{margintocdepth}{\sectiontocdepth}


\newcommand{\showlocaltoc}{%
    \begingroup%
    \def\par{\let\par\endgraf}%
    \etocsetnexttocdepth{2}%
    \setcounter{margintocdepth}{\subsectiontocdepth}%
    \margintoc[-\kaomarginskipabove]% TODO better fix for vertical spacing?
    \setcounter{margintocdepth}{\sectiontocdepth}%
    \endgroup%
}


% This is a hack to keep all margin notes in the correct side: see https://tex.stackexchange.com/questions/236334/sidenotes-appear-on-the-wrong-side-for-tufte-latex-documents
\usepackage{mparhack}  

% Adri: I added a 0pt white space in the marginnotes so that they are properly placed when used at the beginning of a paragraph
% TODO fix
% \leavevmode
% see https://latexref.xyz/_005csbox-_0026-_005csavebox.html
% https://tex.stackexchange.com/questions/315258/prevent-marginnote-from-creating-extra-paragraph
%\pretocmd{\marginnote}{\makebox[0pt]{ }}{}{}

% Move section numbers in margin
%\newcommand*{\numberinmargin}[1]{%
%	\makebox[0pt][r]{#1\autodot\hskip\marginparsep}}
%
%\renewcommand*{\sectionformat}{\numberinmargin{\textcolor{lightgraycolor}{\thesection}}}
%\renewcommand*{\subsectionformat}{\numberinmargin{\textcolor{lightgraycolor}{\thesubsection}}}

% OR
% Color section number in gray
\renewcommand*{\sectionformat}{\textcolor{maincolor}{\thesection}\enskip}
\renewcommand*{\subsectionformat}{\textcolor{maincolor}{\thesubsection}\enskip}

% Change style of the headers (chapter and section titles)
\renewcommand*{\chaptermarkformat}{\textbf{\chapapp~\thechapter}\enskip\color{maincolor}}
\renewcommand*{\sectionmarkformat}{\textbf{\thesection}\enskip\color{maincolor}}

% Call TOC "Table of Contents" instead of "Contents"
\addto\captionsbritish{% Replace "english" with the language you use
	\renewcommand{\contentsname}%
	{Table of Contents}%
}

% TOC Style of Parts (add color)
\newcommand\tocpartstyle[1]
{\scshape\large\bfseries\textcolor{maincolor}{#1}}
\DeclareTOCStyleEntries[pagenumberwidth=2.5em, entryformat=\tocpartstyle]{tocline}{part}%

% Rename Listing to Algorithm
\renewcommand{\lstlistingname}{Algorithm}% Listing -> Procedure
\crefname{listing}{algorithm}{Algorithm}

% ===================================================================
% OTHER (ordered here for special reasons)

\usepackage{csquotes}				% English quotes
\usepackage[begintext=``, endtext='']{quoting}  % to put marks when quoting
\SetBlockEnvironment{quoting}  % TODO indent first does not work
\SetBlockThreshold{0} 

\usepackage{scrhack}
\usepackage{xurl}					% Allow line-breaks in URLs (loaded after biblatex)

\usepackage{suffix}

% ===================================================================
% INPUT

\input{preamble/hyphenations}	% Custom hyphenations
% ===================================================================
% MATH
% ===================================================================
\newcommand{\verylongrightarrow}{\xrightarrow{\hspace*{1.5cm}}}
\newcommand{\explainmath}[1]{\ensuremath{&&{\footnotesize\text{#1}}}} % Only works in align environments

% punctuation of equations, see for example the first answer in
% https://www.reddit.com/r/LaTeX/comments/5xnzg7/correct_grammar_for_putting_an_equation/
\newcommand{\equationPunctuation}[1]{\,{#1}}

% ===================================================================
% REFERENCES
% ===================================================================

% https://davidyat.es/2016/07/27/writing-a-latex-macro-that-takes-a-variable-number-of-arguments/
% \newcommand{\subfigref}[1]{\textbf{(\subref{#1}\subfigrefchecknextarg)}}
% \newcommand{\subfigrefchecknextarg}{\@ifnextchar\bgroup{\subfigrefgobblenextarg}{}}
% \newcommand{\subfigrefgobblenextarg}[1]{, \subref{#1}\@ifnextchar\bgroup{\subfigrefgobblenextarg}{}}

\newcommand{\subfigref}[1]{\textbf{(\subref{#1})}}

% ===================================================================
% COMMON ABBREVIATONS


\makeatletter
\newcommand{\PhD}{Ph.D.\@\xspace}
\makeatother
\newcommand*{\citeeg}{\textit{e.\nobreak\hairsp{}g.},}		% Special \eg command used in citations (doesn't result in weird space)
\newcommand*{\cf}{cf.\@\xspace}
\newcommand{\versus}{\textit{v.\nobreak\hairsp{}s.}\xspace}
\newcommand{\randomvariable}{\textit{r.\nobreak\hairsp{}v.}\xspace} % rv is already used
\newcommand{\randomvariables}{\textit{r.\nobreak\hairsp{}v.\nobreak\hairsp{}s}\xspace} % rv is already used
\newcommand{\rhs}{right-hand side\xspace}

\usepackage{nth}						% For things like 2nd, 3rd, etc.
\usepackage{xstring}

\let\oldnth\nth
\renewcommand{\nth}[1]{%
    \IfInteger{#1}{\oldnth{#1}}{\hbox{#1-th}}%
%    \hbox{$#1$\ifnum1=0#1\relax st\else\ifnum2=0#1\relax nd\else\ifnum3=0#1\relax rd\else th\fi\fi\fi}
}  % it does the job

% ===================================================================
% COLORED ITEMS

% Coloured dots
\DeclareRobustCommand{\colordot}[1]{%
	\begin{tikzpicture}[baseline=(a.south)]
			\node[circle, scale=0.75,color=white, fill=#1] (a) {};
	\end{tikzpicture}%
}

% Coloured square
\DeclareRobustCommand{\colorsquare}[1]{%
	\begin{tikzpicture}[baseline=(a.south)]
			\node[rectangle, scale=0.9,color=white, fill=#1] (a) {};
	\end{tikzpicture}%
}

% Coloured line
\DeclareRobustCommand{\colorline}[1]{%
	\begin{tikzpicture}
		\raisebox{1.5pt}{
			\draw[#1,solid,line width=1.5pt] (0,0) -- (1em,0);
		}
	\end{tikzpicture}%
}

% Coloured arrow
\DeclareRobustCommand{\colorarrow}[1]{%
	\begin{tikzpicture}
		\raisebox{2.5pt}{
			\draw[#1, thick, -stealth] (0,0) -- (1em,0);
		}
	\end{tikzpicture}%
}

% Coloured shade
\newcommand{\colorshade}[1]{\textcolor{#1}{\ding{122}}}


%%% Local Variables:
%%% mode: latex
%%% TeX-master: "../thesis"
%%% End:
	% General Abbreviations, code envs, etc.
\input{preamble/math_commands} % Math Commands from the Deep Learning book

% ===================================================================
% HOTFIXES
\renewcommand*{\figureformat}{%
  \figurename~\thefigure%
  % \autodot% DELETED
}
\renewcommand*{\tableformat}{%
  \tablename~\thetable%
  % \autodot% DELETED
}



% Redefine the paragraph environment to add a period at the end (yes, I am lazy)
% from here: https://tex.stackexchange.com/questions/328355/add-a-period-after-each-paragraph-title
%\makeatletter
%\NewCommandCopy\latexparagraph\paragraph
%\RenewDocumentCommand{\paragraph}{sO{#3}m}{%
%  \IfBooleanTF{#1}
%    {\latexparagraph*{\maybe@addperiod{#3}}}
%    {\latexparagraph[#2]{\maybe@addperiod{#3}}}%
%}
%\newcommand{\maybe@addperiod}[1]{%
%  #1\@addpunct{.}%
%}
%\makeatother


% This one is from here:
% https://tex.stackexchange.com/questions/685660/automatically-add-a-final-period-if-argument-of-paragraph-does-not-end-with-a
\NewCommandCopy\latexparagraph\paragraph
\RenewDocumentCommand{\paragraph}{sO{#3}m}{%
  \IfBooleanTF{#1}
    {\latexparagraph*{\maybeaddperiod{#3}}}
    {\latexparagraph[#2]{\maybeaddperiod{#3}}}%
}

\ExplSyntaxOn
\NewDocumentCommand{\maybeaddperiod}{m}
 {
  \userninefourtwoninethree_maybeaddperiod:n { #1 }
 }

\cs_new_protected:Nn \userninefourtwoninethree_maybeaddperiod:n
 {
  \regex_match:nnTF { .*[\!\?\.] \Z } { #1 } { #1 } { #1. }
 }
\ExplSyntaxOff

%%% Local Variables:
%%% mode: latex
%%% TeX-master: "../thesis"
%%% End:
